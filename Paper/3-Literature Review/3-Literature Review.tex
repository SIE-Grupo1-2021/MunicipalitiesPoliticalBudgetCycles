\section{Literature Review}
The theoretical framework for Political Budget Cycles (PBC) was formally established by \textcite{nordhaus1975}. In his work, he examined the behavior of democratic political systems when facing choices between present and future welfare. Capitalizing from data showing that voters are rather sensitive to inflation and unemployment, he designed a model to explain and predict budgetary policy decisions made by political authorities. He concludes that incumbent politicians go from austerity early in their term to greater spending in election periods. Similar conclusions were reached by \textcite{rogoff1988} that further developed this model. Emphasizing the importance of temporary information asymmetries, assuming that voters observe government investments the year before from elections, incentives clearly exist for macroeconomic policy manipulation. These theoretical conclusions are also backed by the political-economic equilibrium model proposed by \textcite{drazen2010}. Furthermore, and rather important, all three authors argue that these measures have the potential to deteriorate welfare, meaning that their examination can be greatly valuable.

At a national level, there has been several attempts to seek relationships between fiscal policy and electoral processes. With data from Mexico's government budget by types of expenditure, \textcite{gonzalez2002} found evidence for the existence of PBC in the form of significant raises in infrastructure spending, starting six quarters before elections to then diminish the one after. Additionally, she found mild changes in current transfers, and that the magnitude of the cycle directly depends on the democratic stability at a given time. \textcite{lankaster2017} has similar findings when analyzing macroeconomic fiscal variables in 13 Latin American countries. The evidence she shows is more timid, and argues that the effectiveness of fiscal manipulation is determined by time and the specific sets of economic conditions. With this in mind, they both recommend the examination of this phenomenon at a local government level due to the homogeneity of conditions in terms of political, social, legal and economic context.

With data from Portuguese municipalities, \textcite{veiga2007} in many ways pioneered the study of political budget cycles at a local government level where they found significant increase in expenditure and reduction in taxes, mainly in investment "highly visible to the electorate" and simultaneous reduction in "not visible" spending such as transportation, machinery and equipment. Similarly, \textcite{drazen2010}, using data from Colombian local governments, find evidence for the existence of PBCs using budget composition rather than level of expenditure, particularly in infrastructure related to transportation, water treatment and power plants. Nonetheless, they do not find significant changes in deficits and total expenditure like \textcite{veiga2007} do.

Analyzing budgetary data from Brazil's local governments and Indonesian municipalities, \textcite{sarukai2010} and \textcite{setiawan2017} respectively found evidence that the fiscal surplus of local governments decreases in election years, due to increase in spending and reduction in taxation. Contrary to the findings of \textcite{veiga2007} and \textcite{drazen2010}, they find that investments decline in election years. This means that countries with similar economic developments, may show different results when measuring opportunistic cycles. Additionally, \textcite{sarukai2010} found that each fiscal variable behaves differently around election years and according to the political alignment with higher levels of government. This conclusion is also drawn by \textcite{corvalan2018} when investigating indirect PBC in Chile. Particularly, they observe that central government transfers to municipalities increase during elections, and that this is significantly larger when the incumbents are politically aligned with the national government. Among aligned mayors, the transfer increase is even larger when the margin is relatively tight. With that said, \textcite{chortareas2016} found that the mayor's political alignment had no effect in PBC, when exploring this idea with data from Greece.

The study of PBC has come a long way in terms of the data and variables used in the attempt to further explain its existence. Tenure has been found to have a direct relationship in some countries such as Indonesia \parencite{sarukai2010} but none in others like Greece \parencite{chortareas2016}. Studying data from Italian local governments and mayor's characteristics, \textcite{alesina2018} found that there is an inverse relationship between the mayor's age and magintude of PBC, possibly due to their long term career prospects. Following the framework established by \textcite{nordhaus1975}, \textcite{labonne2016} discovers data suggesting that local governments from the Philippines shift expenditure to boost employment the quarters before elections just for it to decrease after the process is done, which goes in hand with the conclusion reached by \textcite{alfaro2019} that voters use the economic environment as input to assign electoral support. Furthermore, he detects more pronunciation of the cycles when the incumbent’s term is limited and trying to transfer the position to a relative, and less pronounced when there are no challengers. Finally, \textcite{bonfatti2019} encounter evidence showing that fiscal rules, setting caps for spending, have worked to diminish the magnitude of Political Budget Cycles in Italy.

Finally, the effects of fiscal manipulation in the form of Political Budget Cycles on the electoral chances of incumbents or political parties has been studied profoundly. After further inquiring by \textcite{aidt2011}, they came to the conclusion that, in Portugal, incumbents ́ odds of being elected in fact increase when expenditure increases. On top of that, they find an inverse relationship between distortion of spending and winning margin, meaning that the distortion is bigger the tighter the race was. Research by \textcite{drazen2010} and \textcite{cassette2014}, found that voters punish incumbents for the accumulation of debt, with an effect increasing with its level, even when it does seem to be the main explanation for its long term accumulation \parencite{alesina2016}. Short-term effects act in the opposite direction. This reveals that elections may work as "debt brakes", even if mild ones. Particularly, for the case of Costa Rica, \textcite{hernandez2014}, found that inflation is the only macroeconomic variable that has significant prediction power for reelection purposes at a partisan national level. 

