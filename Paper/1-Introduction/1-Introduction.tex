\newpage
\setcounter{page}{1}

\section{Introduction}\label{sec:intro}

Local governments --also called municipalities-- are fundamental institutions for the governance of a country. Their closeness to the community provides them first-hand recognition of their challenges and threats. These institutions handle a significant amount of resources: \textcite{cgr2020} reported that municipalities initially budgeted almost 600 billions of colones. In theory, these funds should be efficiently allocated to maximize social welfare, however, there's evidence of a relationship between fiscal policy and electoral incentives:  spending moves with a predictable pattern when elections-motivated candidates and parties are running for election. We shall call this pattern Political Budget Cycles (henceforth, PBCs). \parencite{chortareas2016,drazen2010}

This paper aims to find if preelectoral politically induced budget cycles exist in voter-attractive expenditures in Costa Rican municipalities and their magnitude. Proving the existence of these cycles has been difficult for two reasons. First, showing a direct link between an increase in expenditure with an increase in incumbent's probability of reelection isn't possible many times because of the lack of disaggregated data on expenditures, primarily, the attractive spending categories. Second, the data are seldom available at the local level for long periods in many countries. 
%Third, the effect of mayors’ characteristics as voters preferences for spending could be correlated with the PBCs, therefore, the effect would be endogenous. 

To overcome these challenges, we bring together a detailed data set with disaggregated expenditures that municipalities must budget and get approval from the Comptroller General of the Republic, and then report the actual spending yearly. Our data set contains three detailed levels of disaggregation for 81 municipalities.\footnote{In 2017, Río Cuarto separated from Grecia leading to the creation of the 82th municipality in the 2006-2020 period, which covers three municipal elections. In our analysis, we exclude Río Cuarto as his mayor was elected in 2020, the last period of our analysis. }  Previous work usually analyzes the first level, that is, remunerations, services, and investment expenditures, while we use the second and third levels that report categories such as publicity, transportation, construction of land communication routes, activities, and others. These types of expenditure are more visible to the voters and will allow us to make a direct connection to the electoral manipulation in the pre-electoral year. This enables us to analyze changes in expenditure comparing the year before elections to the rest of the incumbent's governing period in spending categories used to capture votes. 
%Regarding the endogeneity of spending correlated to the mayors' characteristics, we implement an IV strategy to identify the effect of these characteristics in spending decisions. 

Our analysis proceeds in three steps. First, we present a description of the data. We introduce the data set of municipalities' spending and their levels of disaggregation. Furthermore, we describe the demographic variables of the Costa Rican Security Fund (\textit{Caja Costarricense del Seguro Social}, henceforth CCSS). Then, we compile a new data set on political variables of parties and their candidates across time with information from the Supreme Court of Elections. 

Second, we present the institutional background that regulates Costa Rican municipalities. We present the mechanisms that allow the PBCs to exist. Also, this helps us choose voter-friendly spending before elections. We explain the different types of municipal expenditure. Afterward, we describe the steps for approving spending and the participants of this process, in particular, the functions of the Municipal Council and the Mayor. Next, we describe the elections, which dates are set by the Supreme Court of Elections. Specifically, elections, determined by the Supreme Court of Elections, take place every four years and are exogenous.

In the third and final step, we explain our baseline dynamic panel model to estimate the effects of spending manipulation on its growth in the pre-election year. We choose a dynamic specification to account for the persistence in the spending growth along the years, also, to capture the political and institutional context. Also, we include fixed effects for municipalities. The identification assumption that we rely on is that, by controlling by demographic variables and mayors' characteristics, we isolate all possible shocks and identify the campaign year effect. 

We show that most of the spending categories have persistent growth over time, which is a first sign of the effective regulation from the General Comptroller and a result against the existence of municipal PBCs in Costa Rica. Contrary to theory, we do not find traditional PBCs in most spending categories; nevertheless, we show that there is a decrease in spending in the period prior to the campaign year (two years before casting the ballot) in some categories from 6\% to 35\%. However, we do prove that there are PBCs in protocol activities, which is one of the most discretionary and visual expenditures. Activities spending decreases by 34.8\% in the year before the campaign year and increases by 30.2\% the year before casting the ballots with respect to previous years of administration. Also, we observe there aren't any kind of PBCs in the most constant expenditures like remunerations and rentals. Moreover, we find that the least visible spending categories, such as durable goods, decrease their budget in the year prior to the campaign year. 

In addition, we theorize possible causes of spending concentration in the first years of the incumbent government. We believe that the institutional context with their burocracy, surveilance and regulation is a driver to spend less in the last government's periods, since there's a big chance the final outcome wouldn't be finished by the end of the incumbent's government.  

Our paper contributes mainly to three pieces of literature. First, our work contributes to our understanding of PBC theory. The theoretical framework was formally established by \textcite{nordhaus1975}, by designing a model to explain and predict budgetary policy decisions made by political authorities. He concludes that incumbent politicians go from austerity early in their term to greater spending in election periods. Similar conclusions were reached by \textcite{rogoff1988}, who further developed this model. Emphasizing the importance of temporary information asymmetries, assuming that voters observe government investments the year before from elections, incentives clearly exist for macroeconomic policy manipulation. These theoretical conclusions are also supported by the political-economic equilibrium model proposed by \textcite{drazen2010}. However, to the best of our knowledge, the literature on PBC theory has not considered institutional friction. On top, it hasn't considered spending growth around other years than the election year. 

Second, we contribute to the analysis of PBCs in developing countries. At the national level, several attempts have been made to seek relationships between fiscal policy and electoral processes. \textcite{gonzalez2002} found evidence for the existence of PBC in the form of significant increases in infrastructure spending by the federal government, starting six quarters before the elections, then decreasing one afterward. \textcite{lankaster2017} has similar findings when analyzing macroeconomic fiscal variables in 13 Latin American countries. The evidence she shows is more timid and argues that the effectiveness of fiscal manipulation is determined by time and the specific sets of economic conditions. However, both highlight in their findings that such a phenomenon should be studied at a local authorities' level due to the heterogeneity of the legal, social, and political conditions. 

Finally, this article contributes to our understanding of PBCs at the municipal level. Furthermore, we contribute to the study of this phenomena by using disaggregated data on voter-attractive expenditures. We follow the method employed by pioneers in the study of PBCs such as \textcite{veiga2007} and \textcite{drazen2010}, where they found significant increases in expenditure and reduction in taxes in Portugal and Colombia, respectively, mainly in investments "highly visible to the electorate" and simultaneous reduction in "not visible" spending such as transportation, machinery, and equipment. Additionally, their findings suggest that the composition of the budget fluctuates greatly in infrastructure spending related to transportation, water treatment, and power plants. We take this even further by including very specific budgetary items seldom included in this kind of research such as overtime, activities, rentals publicity and, commercial and financial services. To the best of our knowledge, we are the second paper that analyzes disaggregated spending categories at the municipal level on developing countries. 

The existence of PBCs proves that in a democratic context there is legal room for elected public servants --with a high degree of decision-making power-- to prioritize their career prospects rather than their constituents' present and future welfare. Furthermore, it can show links between politics and socioeconomic outcomes, provide insight on how fiscal policy is conducted, and, more importantly, find practical measures that counteract such manipulation and improve social welfare. \parencite{alesina2018,chortareas2016, corvalan2018, setiawan2017}

This article is divided as follows. Section \ref{sec:institution_background} covers all the financial, legal, and constitutional background of Municipalities in Costa Rica. Section \ref{sec:data} summarizes the data used for our model. Section \ref{sec:empirical_strategy} explains the empirical strategy used to approach our research question. In section \ref{sec:results} we show our results and discusses similarities, differences and possible explanations, in light of literature reviewed. Finally, Section \ref{sec:conclusions} concludes. 

