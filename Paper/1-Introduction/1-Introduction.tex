\newpage
\setcounter{page}{1}

\section{Introduction}

Local governments -also called municipalities- are fundamental institutions for the governance of a country. Their closeness to the community lets them have first-hand recognition of their challenges and threats. These institutions handle a significant amount of resources: \textcite{cgr2020} reported that municipalities initially budgeted almost 600 billions of colones. Even if these funds should be efficiently allocated to maximize social welfare, the study of Political Budget Cycles (henceforth, PBCs) has made the relationship between fiscal policy and electoral incentives rather visible. This field of economic research has progressed greatly in the past decades and has found evidence in several countries that spending moves with a predictable pattern of elections-motivated candidates and parties. \parencite{chortareas2016,drazen2010}

This paper aims to find if pre-electoral politically induced budget cycles exist in voter-friendly expenditures in Costa Rican municipalities and their magnitude. Such phenomenon is rather important to investigate since, in some ways, proves that in a democratic context there is legal room for elected public servants, with a high degree of decision-making power, to prioritize their career prospects rather than their constituents' present and future welfare. Furthermore, it can show links between politics and socio-economic outcomes, induce greater understanding on how fiscal policy is conducted and, more importantly, to practical measures that counteract such manipulation to further boost welfare. Additionally, we intend to further study the relationship of the cycles’ magnitude with (1) incumbent running for reelection and (2) age. \parencite{alesina2018,chortareas2016}

Proving the existence of these cycles in specific spending categories has been difficult because of two different reasons.  First, the availability of disaggregated data for long periods isn’t available in many countries at the local level. Second, the effect of mayors’ characteristics as voters preferences for spending could be correlated with the PBCs, therefore, the effect would be endogenous. 

To answer our question, we analyze a detailed data set with disaggregated expenditures that municipalities are obligated to report to the General Comptroller of Costa Rica. This data set contains three detailed levels of disaggregation for 81 municipalities\footnote{In 2017, Río Cuarto separated from Grecia leading to the creation of the 82th municipality in the 2006-2020 period, which covers three municipal elections. In our analysis, we exclude Río Cuarto as their mayor was elected on 2020, the last period of our analysis. }  The literature usually analyzes the first level (i.e. capital, current and investment expenditures) while we use the second and third levels that report categories like publicity, transportation, construction of land communication routes and others. These types of expenditure are clearly more visible to the voters and will allow us to make a direct connection to the electoral manipulation in the pre-electoral year. Regarding the endogeneity of spending correlated to the mayors' characteristics, we implement an IV strategy to identify the effect of these characteristics in spending decisions. 

Our analysis proceeds in three steps. First, we present a description of the data. We introduce the data set of municipalities' spending and their levels of disaggregation. Also, we describe the socioeconomic variables that helps us control for macroeconomic and climate shocks from the Central Bank of Costa Rica. Also, we compile a new data set on political variables of parties and their candidates across time with information from the Supreme Court of Elections. As well, we use demographic variables from the Costa Rican Security Fund (\textit{Caja Costarricense del Seguro Social}, henceforth, CCSS). 

Second, we present the institutional background that regulates Costa Rican municipalities. It's necessary to explain the mechanisms that allows the PBCs to exist in the first place and second to choose voter-friendly spending before elections. We explain the sources of revenue and the different types expenditure municipalities of municipalities. Also, we describe the steps for approving spending and the participants of this process, in particular, the functions of the Municipal Council and the Mayor. Next, we describe the elections. Specifically, the elections timing is determined by the Supreme Court of Elections every four years and is exogenous. Aditionally, we explain the separation of the National Partisan Elections and the Municipal.  

Third, we explain our baseline dynamic panel model to estimate the effects of spending manipulation to its growth on pre-electoral year. Since continuous reelection for mayors isn't prohibited in Costa Rica, we capture the previously unexplored effects that a prolonged incumbency may have on the magnitude of municipal PBCs. We analyze how much bigger are PBCs for incumbents seeking reelection and how these incentives diminish as they become older. The identification assumption that we rely on is that by controlling by macroeconomic, social, demographic variables and mayors' characteristics, we isolate all possible shocks and identify the proposed effects. 
