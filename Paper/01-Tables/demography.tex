\begin{table}[ht]
\caption{Demographic and Economic Variables per Municipality ($\mathbf{Mun'_{it}}$)}
\begin{center}
    

\resizebox{\textwidth}{!}{
\begin{tabular}{l l l}
\hline
\hline
Variable & Description and measure & Source \\ [0.5ex] 
\hline
%Pop.den & Population density, in average people per km^2 & TSE\\ 
Population $< 15$ & Population under 15 years old, percentage & CCSS \\
Population $> 65$ & Population over 65 years old, percentage & CCSS \\
K-12 & Number of K-12 centers & CCSS \\
%Def.Tot & Deficit as a percentage of total expenditure & CGR \\
%Curr.Trans & Transfers from the Central Govt. for current expenditure & CGR \\
%Cap.Trans & Transfers from the Central Govt. for capital expenditure & CGR \\
%Debt & Total debt hold by Municipalities & CGR \\
%Tot.Exp & Total executed expenditure by Municipalities & CGR 
%Free.Zones & Number of Free Zones & CINDE \\
\hline
\hline
\end{tabular}}
\label{table:demoecon}
\end{center}
\footnotesize
\textit{Note: Table \ref{table:demoecon} presents the demographic and economic controls used in the model for every municipality any given year. Both population variables are given as a percentage of the total amount of residents in the canton. K-12 centers include all education institutions from kindergarten to twelveth grade inside the canton's borders. This information is extracted from the Costa Rica Social Security Institution Actuarial Statistics from 2006 to 2020.}
\end{table}