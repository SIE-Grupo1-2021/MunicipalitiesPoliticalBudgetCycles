\section{Results and Discussion}
	Table 1 shows the results we obtained at the three main levels of expenditure. Before giving the spotlight to the elections variable, it is important to highlight the persistency of spending by local authorities. This can be seen in high coefficients, reaching levels over 0.8 associated with significance levels of 1%. It therefore can be inferred, that current spending greatly depends on the budget assigned to any given item the year before. That being said, even if this smooth pattern may be expected due to the regulation and surveillance that is given to the use of public funds, which by itself may be considered evidence against cycles, we can see that fluctuations do happen when analyzing municipal finances and its different levels of disaggregation.
	The first and most aggregate one consists of remuneration, services and durable goods. The last two show significant cyclical behavior around the 2 years before the election. While services shows a lesser effect with a decrease of 7.13% at a 1% significance level, budget item to which most of the attention is usually placed when studying political budget cycles moves in a rather surprising direction. By this we mean municipal investments shown as durable goods that encompass all machinery and equipment bought or rented and any construction and/or improvements made to the “canton’s” insfraestruture due to its high visibility. Overall with 5% level of significance, we see a decrease of 27.43% two years to the election. 
	Even if this pattern may come as a surprise, mainly due to its unintuitative, relatively unprecedented in most literature and counter theoritical nature, this may imply that PBCs do manifest in Costa Rica at a local level, but in a particular way, were the burocratic system has a prolonged timeframe between budgeting and executing infrastructure projects. If true, this means that decision making officials have incentives to come up with higher budgets only to see the results associated with them being implemented just before or during the campaign period. Moreover, the decrease could then be associated to a lack of interest from the mayor to invest later on in its administration and avoid the risk of having these projects be executed during another one, and therefore loose his or her “credit” over it. 
	Moving on, the level of dissagragetion of the data used allows us to see not only the configurations of budgeting at large but also the movement and magnitude of its smaller components. For example, even if remunerations as a whole shows no evidence of behaving in a cyclical manner, at a 1% level of signficance we detect a 14.88% and 6.27% decreases in overtime and other types of payment, respectively. This pattern could possibly be linked to a process of reducing discretionary expenses to further improve the financial standing of the municipality the year before the elections take place and have more space for “voter-friendly” spending. The fact that this not be reflected in significant coefficients on may be because each municipality uses this excess in very different manners so that it becomes statistically undetectable. 
	The budget items connected to comercial and financial services including publicity and the ones that have to do with social activities and protocol also show compelling results. In a predictable way, we see the first one decrease significantly two years before the election, where the budget for this declines by 20.86%, at a 1% significance level. Additionally, we see a significant growth in activities related budgeting of 35.97% the year before the elections accompanied by a decline of 22.93% the year before that (both at a 1% level of significance). Many possible reasonings could be given as to why this may happen, however we identify three non-exclusive ones: The first is that municipalities can decrease very discretional spending, such as the ones that involve activities (reduction of 34.77% at 1% level of significance), two years before the elections, to give themselves more financial freedom next fiscal year to spend more on visible budgetary items. Second, according to the original theory of PBCs, increased spending before the electoral period would prove ineffective to attract voters. Therefore, the mayors have incentives to reduce spending right up to one year before the polls open. Finally, in a context of high abstencionism, the people who participate in municipal activities are more likely to vote than those who don’t. With this in mind, authorities with the prospects of obtaining political advantage, will definitely be incentivized to spend more on those types of events (30.15% increase in activities’ budget with 1% significance level) shortly before the ballots are emitted to secure the support of highly active contituents.  
	Finally, lets refer to the component of durable goods with significant results: Machinery, equipment and furniture. This element shows a similar pattern as its aggregate item does. At a 5% level of significance, having a decline on its budget of 23.22% two years ahead of the elections. We hypothesize, analogously to investments in general, that this cycle is linked to strategic allocation of resources in a time frame that allows for the results of those projects to be properly seen in a period close enough to the opening of the polls.
	


