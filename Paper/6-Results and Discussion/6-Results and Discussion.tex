\section{Results and Discussion}\label{sec:results}

Table \ref{table:regressions} shows the results we obtained at the three main levels of expenditure. We find persistency of spending by local authorities shown by high coefficients in the dependent variable first lag, reaching levels over 0.8. Therefore, it can be inferred that current spending is highly dependent on the budget assigned to any given item in the year preceding. Even if this smooth pattern may be expected due to the regulation and surveillance that are given to the use of public funds, which by itself may be considered evidence against cycles, we can see that fluctuations do occur when analyzing municipal finances and its different levels of disaggregation.
	
The first and most aggregate is remuneration, services, and durable goods. The last two show significant cyclical behavior around the year prior to the political campaign. Surprisingly, in an unexpected direction, services show a lesser effect with a decrease of 7.13\% and durable goods -- which encompass all machinery and equipment purchased or rented, and any construction and/or improvements made to the canton's infrastructure-- decrease in 27.43\% two years before the election. 

Even if this pattern may come as a surprise, mainly due to its unintuitive and unprecedented nature compared to the results found in the most recent literature \parencite{drazen2010, veiga2007}, and counter-theoretical nature with respect to \parencite{nordhaus1975, rogoff1988}, this may imply that PBCs manifest themselves in Costa Rica at a local level, but in a particular way, where the bureaucratic system prolongs the time frame between budgeting and executing infrastructure projects. Therefore, mayors plan their budget so that the results are public before the change of government to get recognition for the improvements in the communities they serve. This incentives higher budgets early on in their administration to then reduce them shortly after its over. 

\begin{landscape}
    \begin{table}[ht]
\def\sym#1{\ifmmode^{#1}\else\(^{#1}\)\fi}
\caption{Effect of elections on different expenditures}
\begin{center}
\resizebox{1.4\textwidth}{!}{
\begin{tabular}{lcccccccccc}
\hline
\hline
Expenditures                                                                        & $y_{t-1}$ & $y_{t-2}$ & $y_{t-3}$ & $\text{Elec}_t$ & $\text{Elec}_{t-1}$ & AR(1) & AR(2) & Sargan-Hansen test & $N$  & Instruments \\
\hline
0-Remuneration                                                                      & 0.783\sym{***}  &           &           & 0.016           & -0.004              & 0.001 & 0.910 & 0.017              & 1125 & $y_{t-5}$   \\
\hspace{2mm}0.01-Basic Remuneration                                & 0.828\sym{***}  &           &           & 0.018           & 0.007               & 0.000 & 0.756 & 0.023              & 1125 & $y_{t-5}$   \\
\hspace{4mm}0.01.01-Salaries                                       & 0.734\sym{***}  & 0.031     &           & 0.019           & 0.015               & 0.000 & 0.942 & 0.062              & 1042 & $y_{t-6}$   \\
\hspace{2mm}0.02-Contingent Remuneration                           & 0.524\sym{***}  &           &           & -0.018          & -0.014              & 0.000 & 0.245 & 0.014              & 1125 & $y_{t-3}$   \\
\hspace{4mm}0.02.01-Overtime                                       & 0.617\sym{***}  & -0.039    & -0.067    & -0.021          & -0.149\sym{***}           & 0.000 & 0.351 & 0.142              & 925  & $y_{t-6}$   \\
\hspace{4mm}0.02.05-Subsistence allowance                          & 0.549\sym{***}  &           &           & 0.009           & 0.063\sym{***}            & 0.000 & 0.179 & 0.230              & 1123 & $y_{t-2}$   \\
1-Services                                                                          & 0.625\sym{***}  &           &           & 0.016           & -0.071\sym{***}           & 0.000 & 0.163 & 0.538              & 1125 & $y_{t-2}$   \\
\hspace{2mm}1.01-Rentals                                           & 0.479\sym{***}  &           &           & -0.003          & -0.166              & 0.000 & 0.338 & 0.210              & 1080 & $y_{t-2}$   \\
\hspace{4mm}1.01.02-Machines, equipment and mobiliary              & 0.473\sym{***}  &           &           & -0.026          & -0.157              & 0.000 & 0.357 & 0.432              & 1061 & $y_{t-2}$   \\
\hspace{2mm}1.03-Financial and commercial services                 & 0.552\sym{***}  &           &           & 0.029           & -0.209\sym{***}           & 0.000 & 0.075 & 0.303              & 1122 & $y_{t-2}$   \\
\hspace{4mm}1.03.02-Advertisement and Publicity                    & 0.347\sym{***}  &           &           & -0.002          & -0.124              & 0.000 & 0.133 & 0.176              & 886  & $y_{t-2}$   \\
\hspace{2mm}1.07-Training and Protocol                             & 0.368\sym{***}  &           &           & 0.360\sym{***}        & -0.229\sym{***}           & 0.000 & 0.104 & 0.055              & 1092 & $y_{t-5}$   \\
\hspace{4mm}1.07.02-Protocolary and Social Activities              & 0.269\sym{**}   & 0.113\sym{**}    &           & 0.302\sym{***}        & -0.348\sym{***}           & 0.000 & 0.911 & 0.048              & 906  & $y_{t-6}$   \\
\hspace{2mm}1.08-Maintenance and Repairs                           & 0.410\sym{***}  &           &           & 0.004           & -0.071              & 0.000 & 0.141 & 0.843              & 1125 & $y_{t-2}$   \\
\hspace{4mm}1.08.01-Buildings, constructions and lands maintenance & 0.385\sym{***}  &           &           & 0.134           & -0.020              & 0.000 & 0.743 & 0.670              & 860  & $y_{t-2}$   \\
5-Durable goods                                                                     & 0.326\sym{***}  &           &           & -0.133          & -0.274\sym{**}            & 0.000 & 0.674 & 0.135              & 1125 & $y_{t-2}$   \\
\hspace{2mm}5.01-Machinery, Equipment and Mobiliary                & 0.103\sym{**}   &           &           & -0.197          & -0.232\sym{**}            & 0.000 & 0.505 & 0.828              & 1121 & $y_{t-2}$   \\
\hspace{2mm}5.02-Construction, Additions and Remodelating          & 0.366\sym{***}  & 0.097     &           & -0.116          & -0.296\sym{**}            & 0.000 & 0.493 & 0.118              & 956  & $y_{t-2}$   \\
\hspace{4mm}5.02.02-Roads                                          & 0.499\sym{***}  & 0.153\sym{***}  &           & -0.119          & -0.177              & 0.000 & 0.690 & 0.390              & 787  & $y_{t-3}$  \\
\hline
\hline
\end{tabular}}
\end{center}
\label{table:regressions}
\footnotesize
\textit{Note: The table \ref{table:regressions} presents results of estimating Eq.(\ref{eq:main_specification}) with the \textcite{blundell1998} system GMM estimator. Each row corresponds to a different regression, where the dependent variable is the log real per capita expenditure accordingly. The first three columns shows the coefficient associated with the first, second and third lag of the dependent variable. The following two show the coefficients associated to the elections year dummy and its lag. Then, we show, the results of the Arellano-Bond test for first and second order autocorrelation. $H_0:$ the first-differenced residuals have autocorrelation of order $k$. Also, the Sargan-Hansen test result. $H_0:$ overidentifying restrictions are valid. Finally, the number of observations, and the last lag of dependent variable used as an instrument. Robust standard errors in parentheses with finite-sample correction for the two step covariance matrix as developed by \textcite{windmeijer2005}. Instruments collapsed as suggested by \textcite{roodman2009}.   $\sym{***}$ significant at least at $1\%$, $\sym{**}$ at least at $5\%$, $\sym{*}$ at $10\%$. }
\end{table}


\end{landscape}

Moving on, the level of dissagregation of the data used allows us to see budget's allocation at the municipal level but also the movement and magnitude of its smaller components. For example, even if the remuneration's category as a whole shows no evidence of behaving in a cyclical manner, we detect a decrease of 14.88\% and 6.27\% in overtime and other types of payment, respectively. This pattern could possibly be related to a process of reducing discretionary expenses to further improve the financial standing of the municipality the year before the elections and have more space for voter-friendly spending. The fact that this is not reflected in a general pattern at the country level could be due to heterogeneity in spending changes across municipalities in the campaign year. 

The spending categories related to commercial and financial services, including publicity and social activities and protocol, also show compelling results, especially when noticing that there are not studied often in literature \parencite{chortareas2016, drazen2010, veiga2007}. Predictably, we see that the first decreases by 20.86\% two years before the election. Furthermore, we see a significant growth in activities-related budgeting of 35.97\% the year before the elections, accompanied by a decline of 22.93\% the year before that. There may be multiple possible explanations; however, we identify three nonexclusive ones. The first is that municipalities can decrease very discretionary spending, such as those that involve activities (reduction of 34.77\%), two years before the elections, to give themselves more financial freedom next fiscal year to spend more on visible budgetary items. Second, according to the original theory of PBCs, increased spending before the electoral period would prove ineffective to attract voters. Therefore, mayors have incentives to reduce spending up to one year before the polls open. Finally, in a context of high abstentionism, the people who participate in municipal activities are more likely to vote than those who don’t. With this in mind, authorities with the prospects of obtaining political advantage, will definitely be incentivized to spend more on those types of events (30.15\% increase in activities’ budget) shortly before the ballots are emitted to secure the support of highly active contituents.  

Finally, let us refer to the component of durable goods with significant results: Machinery, equipment, and furniture. This element shows a similar pattern as its aggregate item does: it decreases 23.22\% two years before the elections. This appears to be the only investment that the municipality makes in which we identify a pattern throughout the studied period, contrary to the findings of \textcite{veiga2007}, \textcite{drazen2010}, and \textcite{chortareas2016}. We hypothesize, analogously to investments in general, that this cycle is linked to strategic allocation of resources in a time frame that allows for the results of those projects to be properly seen in a period close enough to the opening of the polls.
	


