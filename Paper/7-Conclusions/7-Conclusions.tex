\section{Conclusions}\label{sec:conclusions}

We present an approach to Political Budget Cycles with the innovation of using unprecedented disaggregated data seen with rather specific budget items such as rentals, activities, publicity, among others. When reviewing the results, we conclude that none of our hypothesis proved to be correct. Even if we did not find, in most cases, evidence for increased spending the year before the elections, following what the literature reviewed consistently finds, there is a pattern that can be appreciated with the data. With the exception of "activities" and "Training and Protocol", we observe a significant decrease in spending two years before the elections in a plethora of budgetary elements. 

We examine our findings under the scope of the Costa Rican local governments conditions and came up with two main theories as to why they behave the way they do. First, we observe the two items that show a significant decrease and increase, two and one year before the elections, respectively: "Training and Protocol" and its component "activities". In a context of high abstentionism, mayors may have the incentive to save spending in social events for the community until elections are close, specially since residents that participate in those may have a higher likelihood of casting a ballot. Next, local authorities could possibly reduce spending in somewhat discretionary items two years before the election and have more fiscal liberty to maneuver spending the one before their term possibly ends. Also, due to plausible bureaucratic frictions that prolong the time frame between budgeting and executing important projects that may be considered attractive to the voters, there may be budget accumulation in the first half of the electoral period for the current mayor to get recognition for these projects. 

Even though our results and conclusions show possible cyclicity in an alternative manner from the empirical data from other countries may predict, this opens the opportunity to entertain the addition of institutional or bureaucratic frictions in the theory of PBCs, that could possibly produce slight variations to the ones seen thus far. 