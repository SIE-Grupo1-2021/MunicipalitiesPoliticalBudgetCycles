\section{Institutional background}

In 1970 the Municipal Code was approved, this established the structure of local governments and their political control. During the following years, reforms have been set in place to further shape its capabilities and limitations, evolved in the different communities. \parencite[p. 10]{alfaro2009} This event set the municipalities in Costa Rica as the entities in charge of the government and administration of "cantonal" interests and services, under the understanding that the cantón is a figure of geographic division defined within the Costa Rican legal framework. \parencite[Art. 3]{al1998} Also, these institutions can "invest public funds with other municipalities and institutions of the Public Administration for the fulfillment of local, regional, or national purposes, or for the construction of public works of common benefit, following the agreements signed for this purpose." (Art.3) This means that municipalities may have an impact outside their jurisdiction as well. 

The Municipalities in Costa Rica enjoy a high degree of autonomy in administrative and financial affairs. To promote the development of their community, they have managerial freedom over budget administration, provision of certain public services, and the approval of rates, prices, taxes and contributions, among many other things. These capacities should enable a municipality to promote the development of their community. (Art. 4) 

Regarding the internal organization of this institution, the two figures of authority are (1) the municipal council and (2) the mayor. The mayor is the official in charge of the functions inherent to the condition of the general administrator: overseeing the organization, operation, coordination, and faithful compliance with municipal agreements, laws, and general regulations. Additionally, the mayor is responsible for the municipal development plan, which is then presented to the municipal council, along with the ordinary and extraordinary budgets. (Art. 17)

Secondly, the municipal council is an entity composed of councilors, elected under popular election, just like the mayor. The council decides the policy and priorities of the municipality, in addition to defining and approving the municipal budget presented by the mayor. (Art. 13) At this point, it is evident that the mayor fulfills (together with other officials, such as the vice mayors) a similar function to the executive powers in presidential democratic societies, and, the municipal council exercises a function like that of congresses.

An extremely important characteristic of these two prominent entities is that both, the mayor and the councilors (council members), have indefinite re-election. (Art. 15) In fact, this applies to all popularly elected positions in Costa Rican municipalities, which is against the principle of power alternation. 

Note that these entities have governance and power over the budget that allows them to manipulate it: The mayor plans and presents it, while the council acts as a political counterweight that proposes, promotes and approves modifications. This framework permits the manipulation of the municipal budget, motivated by the rational and opportunistic use of resources on behalf of mayors in order to be reelected. This budget pattern refers to the possible existence of Political Budget Cycles.

In order to go further in a PBC study, it requires a clear identification of the flow of resources in the Costa Rican municipalities, explained in table N.1. (to be added)

The municipalities have two main sources of revenue. First, they collect and administrate various taxes assigned to them (Art 77); which usually represents the biggest proportion of income. Moreover, municipalities have certain advantages in comparison to other public institutions, due to several tax exemptions they have received in the past years, like being excluded from the application of the fiscal rule. The second main source is the income derived from municipal services they provide, such as public lighting, road maintenance, trash management and recycling, etc. (Art. 83) Therefore, it is possible to identify that the current revenues have the most weight in the flow of municipal incomes.

Municipalities also have access to credit under some supervision (Financing). The municipal code establishes that municipalities may be financed in various ways: (1) through loans between municipalities, (2) through the issuance of bonds, or (3) through the issuance of municipal titles. This indebtedness must be reflected in subsequent budgets.

Regarding spending, this must be annually planned and should promote the efficient and equitable distribution of resources. (Art. 101) The table above shows how the different types of expenses are classified. Some of them could present a stronger fluctuation, such as capital expenditure, since this includes road expenses, for example. Other expenses such as financial assets or service expenses are expected to remain rather stable over time.

Having said that, it is important to identify two fundamental facts that should be taken into consideration in this research, since they could influence the behavior of PBCs in Costa Rican municipalities: First of these is the change in the electoral period in Costa Rica. Before the 2016 election, the municipal elections were divided, the election of the councilors was presented simultaneously in February (along with the national elections). On the other hand, the mayoral election took place the same year in December. It was defined by the congress that, in order for the municipalities to achieve greater autonomy and give more relevance to the municipal elections, local authorities will be elected in midterm elections, two years after the national ones. The renewal of all these positions would be carried out every four years as stated in the Electoral Code. \parencite[Art. 150]{al2009} To this end, it was necessary to extend the 2010 electoral period for two years, becoming the only municipal period in history to last for 6 years (2010-2016). Secondly, we consider the emergence of micro-parties (or municipal parties). Their inscription and functions are limited to the municipal one, therefore they can't interfere in national politics in a direct manner. In the last two decades, the country has observed an emergence of new local leadership, supported by greater decentralization, the crisis of the traditional national parties, the new tendencies to citizen participation and concerns regarding the control and poor management of local governments. \parencite[p. 165]{blanco2011} The phenomenon of local parties has been gradually reinforced, especially since the 1998 elections. \parencite[p. 15]{beers2006} With this in mind, it can be argued that the dynamics behind budget management can be greatly dependent on the party's classification. This being national, provincial or municipal. These factors will be evaluated throughout this research. 
