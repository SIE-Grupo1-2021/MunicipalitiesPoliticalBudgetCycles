\section{Institutional background}\label{sec:institution_background}

The Costa Rican Municipal Code was approved in 1970 establishing the structure of local governments and their political control. Over the years, reforms have been made to further shape its capacities and limitations. \parencite[p. 10]{alfaro2009} This event established municipalities as entities in charge of the government and administration of cantonal interests and services, under the understanding that the canton is a figure of geographic division defined within the legal framework of Costa Rica. \parencite[Art. 3]{al1998} Also, these institutions can "invest public funds with other municipalities and organizations of the Public Administration for the fulfillment of local, regional, or national purposes, or for the construction of public works of common benefit, following the agreements signed for this purpose." (Art.3), which means that municipalities can also have an impact outside of their jurisdiction. 

Costa Rican municipalities enjoy a high degree of autonomy in administrative and financial affairs. To promote the development of their community, they have managerial freedom over budget administration, provision of certain public services, and the approval of rates, prices, taxes, and contributions, among many other things. (Art. 4) 

Regarding the internal organization of this institution, the two figures of authority are (1) the Municipal Council and (2) the Mayor. The mayor is the official in charge of the functions inherent in the condition of a general administrator: overseeing the organization, operation, coordination, and faithful compliance with municipal agreements, laws, and general regulations. Additionally, the mayor is responsible for creating the municipal development plan, which is presented to the municipal council, along with the ordinary and extraordinary budgets. (Art. 17)

Additionally, the municipal council is an entity composed of publicly elected councilors, just as the mayor. The council decides the policy and priorities of the municipality, in addition to defining and approving the municipal budget presented by the mayor. (Art. 13) At this point, it is evident that the mayor fulfills (with other officials, such as the vice mayors) a similar function to the executive powers in presidential democratic societies, and the municipal council exercises a function like that of congresses. An extremely important characteristic of these two prominent entities is that both the mayor and the councilors (council members) have no reelection limitations as of 2021. (Art. 15) In fact, this applies to all popularly elected positions in Costa Rican municipalities, which proves to be an obstacle the principle of power alternation. 
Note that these entities have the necessary governance and power over the budget that allow them to manipulate it. The mayor plans and presents it, while the council acts as a political counterweight that proposes, promotes, and approves modifications. This framework allows manipulation of the municipal budget that may be motivated by the rational and opportunistic use of resources on behalf of mayors in order to be reelected. This budget pattern refers to the possible existence of PBCs.

%In order to go further in a PBC study, it requires a clear identification of the flow of resources in the Costa Rican municipalities, explained in table N.1. (to be added)
Although mayors have the freedom to allocate their resources, some public institutions and regulations are supposed to guarantee the correct and legal application of these expenditures. The main external institution involved in municipal budget procedures is \textit{Comptroller General of the Republic} (henceforth CGR). This entity oversees the approval and constant supervision of the finances of these institutions. In particular, the CGR should (1) examine, approve or disapprove the municipalities' budgets, as well as supervise the execution and budget liquidation, (2) supervise that the budgets are organized and formulated in accordance with legal and technical provisions, and, in addition, (3) has the power to determine requirements, procedures, and conditions to make modifications to the budgets, as well as dictate policies, technical manuals, and mandatory compliance guidelines in their jurisdiction. \parencite[p.2]{asamblea2008}

The mayor, the council, and the CGR are involved in budget development following these steps: (1) formulation, (2) approval, (3) execution, (4) control, and (5) evaluation. First, the municipality must \textbf{formulate} the budget in line with the \textit{"operational planning that is carried out in accordance with the medium and long-term plans and the institutional policies and objectives defined for the period."} \parencite[p.3]{asamblea2008} The budget must also comply with certain basic principles that the law implores, such as the principle of universality and integrity, financial management, among others. In addition, the CGR will ensure that the principles of participation, flexibility, sustainability, etc. are respected. The budget must be planned annually, considering that the regular budget is valid from January 1 to December 31. The CGR will rule that certain financial and technical criteria are met, such as at least 20\% must be allocated to health care and no more than 40\% can be allocated to cover administrative expenses. The final step at this stage is that each generated expense must be carefully explained, providing information on its origin and application.

Second, the budget is \textbf{approved}. This point is divided into two phases. First, the municipal council and the mayor internally approve the budget, where the council acts as a counterweight to the mayor and his team by modifying the budget when necessary. In the second place, comes the external approval by the CGR after verifying that it meets the pertinent requirements. This process guarantees two aspects: the internal approval limits the possible imposition of mayor's political interests. The council's scrutiny depends on its composition: a higher concentration of the official party is often associated with a budget more at the whim of the mayor and vice versa. The second control channel is the CGR that enforces the planning, public administration, and legal guidelines. 

Third, the process continues with the \textbf{execution} which corresponds to: \textit{"administrative activities and financial economic operations that allow the perception of income and its use in budgeted expenses, in order to achieve the goals and objectives indicated."} \parencite[p.6]{asamblea2008} On this step, there is intermittent surveillance by audits when it is considered necessary. Generally, it will depend solely on the administrative body to verify that the corresponding standards are met. Finally, the \textbf{control} and \textbf{evaluation} phases come, where there are certain mechanisms that guarantee even more clarity within the process. 

This process of specifying the budget is even more bureaucratic than described. The intervention of the CGR, in addition to the two approval control channels, should generate enough control to stop possible irregularities in the budget. Bureaucracy could counteract PBCs. There are many control mechanisms that, together with external actors (such as the media), should stop imminent measures of budget manipulation. 

%The municipalities have two main sources of revenue. First, they collect and administrate various taxes assigned to them (Art. 77); which usually represents the biggest proportion of income. Moreover, municipalities have certain advantages in comparison to other public institutions, due to several tax exemptions they have received in the past years, like being excluded from the application of the fiscal rule. The second main source is the income derived from municipal services they provide, such as public lighting, road maintenance, trash management and recycling, etc. (Art. 83) Therefore, it is possible to identify that the current revenues have the most weight in the flow of municipal incomes.

%Municipalities also have access to credit under some supervision (Financing). The municipal code establishes that municipalities may be financed in various ways: (1) through loans between municipalities, (2) through the issuance of bonds, or (3) through the issuance of municipal titles. This indebtedness must be reflected in subsequent budgets.

Regarding spending, this must be annually planned and should promote the efficient and equitable distribution of resources. (Art. 101) The table \ref{table:expenditures} shows how the different types of expenses are classified. Some of them could present a stronger fluctuation, such as capital expenditure, since this includes road expenses, for example. Other expenses such as financial assets or service expenses are expected to remain relatively stable over time.

There are four fundamental facts that could influence the behavior of PBCs in Costa Rican municipalities. First, the celebration of the electoral period in Costa Rica changed its dates. Before the 2016 election, the municipal elections were divided, the election of the councilors was presented simultaneously in February (along with the national elections); while the mayoral election took place the same year in December. Congress considered that in order for municipalities to achieve greater autonomy and give more relevance to the municipal elections, local authorities would be elected in midterm elections two years after the national ones. The election of these positions would take place every four years, as stated in the Electoral Code. \parencite[Art. 150]{al2009} To accomplish this, it was necessary to extend the 2010 electoral period by two years, becoming the only municipal period in history to last 6 years (2010-2016). 

Second, we consider the emergence of micro-parties (or municipal parties). Their inscription and functions are limited to municipal politics; therefore, they cannot interfere in national politics directly. In the last two decades, the country has observed an emergence of new local leadership which is supported by: greater decentralization, the crisis of the traditional national parties, the new tendencies to citizen participation and concerns regarding the control and poor management of local governments. \parencite[p. 165]{blanco2011} The phenomenon of local parties has been gradually reinforced, especially since the 1998 elections. \parencite[p. 15]{beers2006} It can be argued that the dynamics behind budget management can be greatly dependent on the party's classification: national, provincial, or municipal. 

Third, there is a growing trend towards politically fragmented local governments. This follows directly from the introduction of new parties (as we mentioned earlier, especially those of municipal competence only) that are beginning to consolidate in the local government policy. \textcite{blanco2011} mentions that: \textit{"In a significant number of cantons, there are fragmented local governments with a predominant but not a majority party"}. \pnotecite[2]{blanco2011} 

Fourth, the cases of corruption uncovered in 2021 associated with the public budget have not exempted municipalities. On November 2021, the Costa Rican authorities arrested 6 mayors, five of them members of the same political party, linked to cases of corruption in the public works tender. \parencite{molina2021} These events have had a direct impact on the analysis period, as they are events of embezzlement of public funds that came from many years ago. But more importantly, these events highlight the importance of constant surveillance and control of public spending by different sectors of our society, which would allow an early detection of irregularities like manipulation of the public finances on municipalities. All of these factors will be evaluated throughout this paper. 
