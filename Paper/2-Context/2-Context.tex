\section{Context}

%Research question


%Motivation and context

%% Relevance

%% Bibliographical Evidence

%% Contribution


% Objectives

% Hypotheses

\begin{hyp}\label{hyp:first}
	There are Political Budget Cycles in Municipalities' different types of expenditures. 
\end{hyp}

\begin{hyp} \label{hyp:second}
	If the PBCs exists, majors seeking reelection spend more on voter-friendly types of expenditure. 
\end{hyp}

\begin{hyp} \label{hyp:third}
	If majors seeking reelection spend more on voter-friendly types of expenditure, this strategy increases their probabilities of winning the election. 
\end{hyp}

\newpage

Local governments -also called municipalities- are fundamental institutions for the exercise of the governance of a country. Their closeness to the community, and therefore, the first-hand recognition of challenges and threats, makes them a state apparatus of great need for the proper exercise of politics.

These institutions handle a significant amount of resources, given that, from an economic point of view, they must be observed very carefully to control the proper management of them, based on the fact that this resources are scarce.

It is extremely curious that, despite the importance of this institution for the economy, the literature on the study of municipalities in Costa Rica, specifically, on the use of the resources they have at their disposal, is extremely scarce - not to mention that null-, this will be observed in greater detail later in this document.

The importance of this institution is highlighted in the Municipal Code, which states that "The municipality has a political, administrative and financial autonomy conferred by the Political Constitution" (Art. 4, Municipal Code). In general, this article has the consequence that local governments have discretion in the use, administration and execution of the resources that are assigned by law.

At this point it may be important to clarify which are, specifically, the resources that are mentioned. To conceptualize the flow of income and expenses of a municipality, see Annex 1.

Public administration is an issue of utmost importance for the economy, especially because it is a current government using  the resources of a community for the benefit of it - in essence it is the exercise of efficient allocation-. For that reason, there have always been questions about the proper use and manipulation of public funds. On the revenue side, there is little a municipality can do to manipulate it beyond the role of financing. On the side of expenditures, there are some rigid items, such as payment of the municipal structure, but some others more discretionary, such as spending on durable goods.

A vox populi is that municipal governments, outside of maximizing social welfare, seek to allocate expenses with a certain convenience, particularly in the face of the electoral contest. This is thanks to the fact that indefinite reelection is allowed within the municipalities (Art. 14, Electoral Code). Apparently this feeling is not exclusive to Costa Rica, and from this, an important range of studies has been generated on what is known as Political Budget Cycles, which raises the study of the municipal budget in order to find some cyclicality, and then, try to identify variables that support said cycle. This literature is explained below.


\textbf{Annex 1}

\begin{table}[]
	\begin{tabular}{cll}
		\multicolumn{3}{c}{Composition of municipal income and expenses} \\
		\multicolumn{1}{l}{} &  &  \\ \hline
		\multicolumn{3}{|c|}{\textbf{Incomes}} \\ \hline
		\multicolumn{1}{l}{} &  &  \\ \hline
		\multicolumn{1}{|c|}{\textit{Current income}} & \multicolumn{1}{c|}{\textit{Capital income}} & \multicolumn{1}{c|}{\textit{Financing}} \\ \hline
		Tax revenue & \multicolumn{1}{c}{Assets sales} & \multicolumn{1}{c}{Internal financing} \\
		Non-tax revenue & \multicolumn{1}{c}{Recovery and advances for public utility works} & \multicolumn{1}{c}{External financing} \\
		Current transfers & \multicolumn{1}{c}{Capital transfers} & \multicolumn{1}{c}{Resources of previous validity} \\
		& \multicolumn{1}{c}{Other capital income} & \multicolumn{1}{c}{Resources of monetary issue} \\
		\multicolumn{1}{l}{} &  &  \\ \hline
		\multicolumn{3}{|c|}{\textbf{Expenses}} \\ \hline
		\multicolumn{1}{l}{} &  &  \\
		\multicolumn{3}{c}{Remuneration} \\
		\multicolumn{3}{c}{Services} \\
		\multicolumn{3}{c}{Materials and supplies} \\
		\multicolumn{3}{c}{Interest and commissions} \\
		\multicolumn{3}{c}{Financial Assets} \\
		\multicolumn{3}{c}{Durable assets} \\
		\multicolumn{3}{c}{Capital transfers} \\
		\multicolumn{3}{c}{Current transfers} \\
		\multicolumn{3}{c}{Amortization} \\
		\multicolumn{3}{c}{Special accounts} \\
		\multicolumn{1}{l}{} &  &  \\
		\multicolumn{3}{c}{Source: Authors' own creation based on information from CGR (2020)}
		
	\end{tabular}
\end{table}

