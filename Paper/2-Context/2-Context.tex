\section{Context}

%Research question

\subsection{Research question}

This paper aims to analyze pre-electoral politically induced budget cycles in Costa Rican municipal fiscal policies. In addition, it examines how these electorally motivated budgetary policies in voter-friendly expenditures affect mayors' reelection prospects. 

%Motivation and context
\subsection{Motivation and context}
%% Relevance

Local governments -also called municipalities- are fundamental institutions for the governance of a country. Their closeness to the community lets them have first-hand recognition of their challenges and threats. This makes them a state apparatus of great need for the proper exercise of politics. The importance of this institution relies in the political, administrative and financial autonomy of the local governments conferred by the Political Constitution" which gives the local governments discretion in the use, administration and execution of the resources that are assigned by law. \parencite[Art. 4]{al1998}  These institutions handle a significant amount of resources: \textcite{cgr2020} reported that municipalities initially budgeted almost 600 billions of colones to spend autonomously in favor of their communities. Also, the scarce resources have to be efficiently allocated in many types of expenditure: social, infrastructure and others. 

For that reason, there have always been questions about the proper use and manipulation of public funds. However, there may be concerns about the existence of mechanisms that truly allow this and not be related to shocks or majors' inexperience executing projects, but, we identify various mechanisms that could allow it to happen: on the revenue side, there is little a municipality can do to manipulate it beyond the role of financing or promoting tax payments; while on the side of expenditures, there are some rigid types of expenditure, such as payment of the municipal structure, but some others more discretionary, such as spending on durable goods.

Nevertheless, despite the importance of this institution for the country's economy, the literature on  study of municipalities in Costa Rica, specifically, on the allocation of  resources they have at their disposal is extremely scarce -almost non-existent-. 

A \textit{vox populi} is that municipal governments, outside of maximizing social welfare, seek to allocate expenses with a certain convenience, particularly when facing the electoral contest. This is a common objective amongst majors given that reelection is allowed within the municipalities as is stated in the Municipal Code \parencite[Art. 14]{al1998}. Apparently, this sensation is not exclusive to Costa Rica: multiple studies have been generated on Political Budget Cycles identify their determinants in Partisan, Municipalities and National Elections, yet, in Central America, this literature focused is quite limited.

%% Bibliographical Evidence

%% Contribution

 Among the very few papers that have explored this issue in the region, one that can be highlighted is Valeria Lankaster paper that analyzes PBC’s in Costa Rica. In this, she finds some econometric evidence in favor of the hypothesis using the general budget of the government in the period 1990-2010. However, she explains that with the aggregate data that she works with, it is extremely challenging to establish concrete conclusions on the matter. Having said that, the exploration of this phenomenon at the level of local governments is practically non-existent in this neighborhood of countries. Going even further, with the exception of papers that work with budget data from Portugal and Colombia, the use of microdata such as specific expense items, is very scarce. 

With all of this in mind, in this paper we intend to traverse into the possible existence of PBC’s within local governments (municipalities) in Costa Rica. With the use of desegregated microdata, we will see if patterns of specific expense items and their relationship with elections can be identified. Furthermore, we aspire to factor in political elements such as the major’s time in office and reelection to build more robust connections between fiscal policy-making and personal aspirations of local authorities. Finally, we will attempt to estimate the effect that “voter-friendly” spending may have on the chances of being re-elected. By following these steps, the plan is to find econometric evidence, either for or against, the possibility that discretionary expenditure is done, in part, with the objetive of achieving personal political goals instead of optimizing the communities’ well-being.


% Objectives

\subsection{Objectives}

\subsubsection{General objective}

Analyze the relationship between Political Budget Cycles in voter-friendly expenditures and the probability of reelection for majors in Costa Rican municipalities from 2007 to 2020. 

\subsubsection{Specific objectives}

\begin{enumerate}
	\item Identify the existence of Political Budget Cycles in Costa Rican municipalities. 
	\item Calculate the change in expenditures the pre-electoral year with respect to the rest of government term. 
	\item Recognize the voter-friendly expenditures that impact the probability of reelection for rerunning majors.
\end{enumerate}

% Hypotheses

\subsection{Hypotheses}

We want to test several hypothesis: 

\begin{hyp}\label{hyp:first}
	There are Political Budget Cycles in Municipalities' different types of expenditures. 
\end{hyp}

\begin{hyp} \label{hyp:second}
	If the PBCs exists, majors seeking reelection spend more on voter-friendly types of expenditure. 
\end{hyp}

\begin{hyp} \label{hyp:third}
	If majors seeking reelection spend more on voter-friendly types of expenditure, this strategy increases their probabilities of winning the election. 
\end{hyp}
