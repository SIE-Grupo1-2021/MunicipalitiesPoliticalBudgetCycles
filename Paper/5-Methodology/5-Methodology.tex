\section{Methodology}

In this section, we present our empirical strategy to study the effects of election years on voter-friendly municipal expenditures. Using the database, we run the following dynamic panel specification considering most of the literature, municipalities' characteristics and the institutional context: 

\begin{equation}\label{eq:main_specification}
	y_{jit} = \sum_{k=1}^{P} \rho_{jit-k} y_{jit-k} + \gamma \text{Elec}_t + \sum_{k=1}^{K} \mathbf{Mun_{it-k}'}\beta_{t-k} + \mathbf{May_{m(i,t)}'}\theta + \mathbf{Nat_{t}'}\alpha + \lambda_i + \varepsilon_{jit}
\end{equation}

where $y_{jit}$ is the log real municipal fiscal variable $j$ for municipality $i$ in year $t$ and $y_{jit-k}$ is the $k$-th lag of the dependent variable used to capture the persistence in the municipal fiscal outcomes. We estimate the model using from one to three lags and we expect the results to be similar among these specifications. We estimate a separate regression for (the log of) each type of government expenditure. That is, each type of government expendi-ture is a differentf.$\text{Elec}_t$ is a dummy which captures the timing of elections. It takes the value of one in periods preceding local elections, and 0 in all others. We set this dummy such that the pre-election period is the year previous to the election if it takes place in the first half of the year and the year of the election, if it is held in the second half. The municipality fixed effect $\lambda_i$ accounts for unobserved characteristic from each municipality and $\varepsilon_{jit}$ an i.i.d. error term.\footnote{Since the Supreme Elections Court established synchronized elections across all municipalities in Costa Rica, we follow \textcite{chortareas2016} not including time fixed effects because the election year effects cannot be separated from aggregate shocks. }  We include additional controls at several levels following the literature and others that fit our institutional context: $\mathbf{Mun_{it-k}'}, \mathbf{May}_{m(i,t)}'}, \mathbf{Nat_{t}'}$. The vector $\mathbf{Mun_{it}'}$ at the municipality's level $i$ in year $t$ controls for demographic variables as the log population, population density, share of population below 15 years old and over 65 years old, number of K-12 centers, the logs of municipality's deficit, current transfers, capital transfers, loans, non-tax income and following \textcite{drazen2010} we control for the total expenditure of the municipality in that year, which will allows us to interpret the coefficient for the political dummy as the election year effect on the share of spending in a given category. We estimate (\ref{eq:main_specification}) with and without controlling for the total expenditure to see the change in levels and the share to analyze this proposed theory. The vector $\mathbf{May_{m(i,t)}'}$ controls for each mayor's characteristics such as age at the start of their government, number of periods as a mayor, gender, incumbent advantage (measured by the share of votes received in that last election), type of political party (municipal, provincial or national) of the incumbent and a dummy variable that takes a value of 1 if the incumbent's party is majority in the Municipal's Council. Finally, the vector $\mathbf{Nat_{t}'}$ contains national variables that don't vary across municipalities. We include Debt-to-GDP Ratio, passive base interest rate, national deficit and log GDP. 

The coefficients of interest $\text{Elec}_t$ dummy and its interactions with reelection dummies and the age of the mayor. In the institutional context, we discussed that, as of 2021, mayors can be reelected indefinitely. Nevertheless, we think that a mayor could influence a Political Budget Cycle in an altruistic manner to make its party's fellows more chances of getting reelected. That's why we consider three reelection variables: (1) if a mayor inscribes herself for the next election, regardless of the political party; (2) if a mayor runs again with the same political party; and (3) if the party runs for reelection in the next period. We expect the coefficients to be statistically significant and positive except the one associated with the age. We would expect that as the candidate get older, the incentives for spending more in the last period to get reelected diminish with the age. 

The specification (\ref{eq:main_specification}) is a standard dynamic panel data one. There are two reasons why standard fixed-effects estimators would be asymptotically biased. First, including a lagged dependent variable and municipality fixed effects renders the OLS estimator biased and inconsistent by the \textcite{nickell1981} bias. Although the Fixed-effects (FE) estimator eliminates the unit specific effects, it cannot eliminate the bias introduced by the inclusion of lagged dependent variables among the regressors, which is correlated by construction with the error term. The order of the FE estimator bias is 1/T, where T corresponds to the time length of the panel. In our case, the time length of our panel is 15 years, consecuently, the use of the Fixed Effects estimator may give rise to a non-negligible bias. To address this concern, we employ the \textcite{blundell1998} two step system GMM estimator for dynamic panel data. This estimator augments the \textcite{arellano1991} difference GMM estimator using lagged differences of the dependent variables as instruments in the levels equations in addition to lagged levels of the dependent variables, which are used as instruments for the equations in first differences. Since the estimated standard errors of the two step GMM estimator tend to be severely downward biased, we correct the bias using the \textcite{windmeijer2005} finite sample correction. There could be misleading results caused by instrument proliferation. In order to make correct inference, we collapse the instrument set, as suggested by \textcite{roodman2009}, to reduce the number of moment conditions. Finally, we perform th \textcite{arellano1991} tests for first-order and second-order serial correlation of the differenced residuals and the Hansen test for over-identifying restrictions. 

The second potential source of bias is the mayor’s age. In the full sample ofmunicipalities, bothOldandAgecould be endogenous in the above equations, even afteraccounting for municipality fixed effects. The reason is that changes in voter preferencesfor spending could be correlated with either changes in the electoral performance of olderor younger candidates—rendering bothOldandAgeendogenous—or changes in the ageprofile of the pool of candidates—renderingAgeendogenous. We address this possibilityby limiting the sample to municipality–years following a close election, where a ‘close’election is decided by a vote margin of 5 percentage points or fewer.12We thus focus ontime variation inOldandAgedriven by arguably idiosyncratic election outcomes ratherthan shifts in voter preferences. We estimate equation (4) in two ways. The first treatsAgeand its interaction as strictly exogenous (conditional on the aforementioned sampleselection), and the second treatsAgeand its interaction as endogenous and usesOldandits interaction as instruments. The variableOldm,tequals 1if the mayor of municipalitymin yeartwas the older of the top two candidates in themost recent election, and 0 otherwise. 

