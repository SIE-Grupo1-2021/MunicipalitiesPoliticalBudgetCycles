\section{Empirical strategy}\label{sec:empirical_strategy}

In this section, we present our empirical strategy for studying the effects of election years on voter-friendly municipal expenditures. Using the database, we run the following dynamic panel specification considering most of the literature, municipalities' characteristics, and the institutional context: 
\begin{equation}\label{eq:main_specification}
	y_{jit} = \sum_{k=1}^{K} \rho_{jt-k} y_{jit-k} + \sum_{k=0}^{1} \gamma_{t-k} \text{Elec}_{t-k} + \mathbf{Municipality_{it}'}\beta + \mathbf{Mayor_{m(i,t)}'}\theta + \lambda_i + \varepsilon_{jit}
\end{equation}

where $y_{jit}$ is the log real municipal fiscal variable per capita $j$ for the municipality $i$ in year $t$ and $y_{jit-k}$ is the $k$-th lag of the dependent variable used to capture persistence in municipal fiscal outcomes\footnote{The description of the dependent variables of our model and the controls can be found in Tables \ref{table:expenditures}-\ref{table:polmayor}. } We estimate a separate regression for (the log of) each type of government expenditure. $\text{Elec}_t$ and its lag are dummies that capture the timing of elections. It takes the value of one in the periods preceding local elections and 0 in all the others. We set this dummy such that the pre-election period is the year previous to the election if it takes place in the first half of the year and the year of the election, if it is held in the second half. This criterion allows us to study the year of the political campaign. The municipality fixed effect $\lambda_i$ accounts for unobserved and constant characteristics from each municipality and $\varepsilon_{jit}$ an i.i.d. error term.\footnote{Since the Supreme Elections Court established simultaneous elections across all municipalities in Costa Rica, we follow \textcite{chortareas2016} by not including time fixed effects because the election year effects cannot be separated from aggregate shocks. }  We include additional controls at several levels following the literature and others that fit our institutional context: $\mathbf{Municipality_{it}'}$ and $\mathbf{Mayor_{m(i,t)}'}$. The vector $\mathbf{Municipality_{it}'}$ at the municipality level $i$ in year $t$ controls for demographic variables such as the share of population under 15 years of age and over 65 years of age and the number of K-12 centers. The vector $\mathbf{Mayor_{m(i,t)}'}$ controls the influence of the political environment on expenditures in each municipality. We include mayors' characteristics like age at the start of their government, gender, incumbent advantage measured by the share of votes of the mayor's party received in the last election at the municipal mayoral level, and the type of political party (municipal, provincial, or national). 

%Finally, the vector $\mathbf{Nat_{t}'}$ contains national variables that don't vary across municipalities. We include national debt-to-GDP and deficit-to-GDP ratios, passive base interest rate and log GDP. 

The coefficients of interest are the $\text{Elec}_{t-k}$ dummies for $k\in\{0, 1\}$ and $y_{ji,t-m}$ for $m\in\{1, 2\}$. We expect the first lag of the dependent variable to be statistically significant and positive. We care about the magnitude of the autocorrelation coefficients in the model specifications with one lag of the dependent variable, as a value closer to 1 indicates high persistence on the expenditure growth, which signals high bureaucracy and strict controls from the Comptroller when approving budgets, vanishing political budget cycles. In the institutional context, we discussed that until 2021 mayors could be reelected indefinitely. Nevertheless, a mayor could influence the Political Budget Cycle in an altruistic manner to make its party's fellows more likely to get reelected if the current mayor does not run for another term. Regarding the election dummies, we expect $\text{Elec}_t$ to be significant and positive and consider the sign of $\text{Elec}_{t-1}$ to be ambiguous \textit{a priori}. If this coefficient is positive and significant, it would indicate that the political budget cycle extends for two years; if it is not significant, it would support the literature regarding the growth in expenditures taking place in the campaign year; and if it is negative and significant, we interpret that the mayors reduce expenditure in preparation for the campaign year. 

%That's why we consider three reelection variables: (1) if a mayor inscribes herself for the next election, regardless of the political party; (2) if a mayor runs again with the same political party; and (3) if the party runs for reelection in the next period. We would expect that as the candidate get older, the incentives for spending more in the last period to get reelected diminish. 

The specification (\ref{eq:main_specification}) is a standard dynamic panel data one. The standard fixed-effects estimator is asymptotically biased. First, including a lagged dependent variable and municipality fixed effects renders the OLS estimator biased and inconsistent by the \textcite{nickell1981} bias. Although the fixed-effects (FE) estimator eliminates the municipalities' specific effects, it cannot eliminate the bias introduced by the inclusion of lagged dependent variables among the regressors, which is correlated by construction with the error term. The order of the FE estimator bias is $O(1/T)$, where T corresponds to the time length of the panel. In our case, the time length of our panel is 15 years; consequently, the use of the Fixed Effects estimator may add non-negligible bias to the coefficients. To address this concern, we employ the \textcite{blundell1998} two-step system GMM estimator for dynamic panel data which augments the \textcite{arellano1991} difference GMM estimator using lagged differences of the dependent variables as instruments in the levels equations in addition to lagged levels of the dependent variables, which are used as instruments for the equations in first differences. Since the estimated standard errors of the two-step GMM estimator tend to be severely downward biased, we correct for the bias using the finite sample correction proposed in \textcite{windmeijer2005}. There could be misleading results caused by instrument proliferation from exploiting all moment conditions in system GMM. To alleviate this concern, we collapse the set of instruments, as suggested by \textcite{roodman2009}, to reduce the number of moment conditions. Finally, we perform the \textcite{arellano1991} tests for the serial correlation of the differenced residuals and the Hansen test for overidentifying restrictions. 

%The second potential source of bias is the mayor’s age. The Age variable could be endogenous in (\ref{eq:main_specification}), even after accounting for municipality fixed effects. The reason is that changes in voter preferences for spending could be correlated with changes in the age profile of the pool of candidates —rendering Age endogenous. We address this possibility by treating Age and its interaction as endogenous. We borrow an instrument from \textcite{alesina2018}: we use the Old variable and its interaction as instruments. The variable $\text{Old}_{it}$ equals 1 if the mayor of municipality $i$ in year $t$ was the older of the top two candidates in the most recent election and 0 otherwise. 

%There could be macroeconomic shocks or weather disasters that affect specific groups of municipalities. This could be due to the culture of certain municipalities or the geography they're placed in. To make our results robust to this shocks, we cluster the standard errors at the municipality level, provincial level, socioeconomic region of planification and economic specialization zones, as in \textcite{alfaro2019}. 

Regarding the model selection algorithm, we mostly follow \textcite{kripfganz2019} and \textcite{kiviet2020}. First, we assume that all variables are predetermined, except for the election dummies, the age and gender of the mayor, and the number of K-12 centers, which we consider exogenous. There is no consensus in the literature on whether or not to include more than one lag in the dependent variable. We apply the described criteria to decide the number of lags of the dependent variable that could be contemporaneous with the error term to choose the optimal specification. We search for the most parsimonious model\footnote{The most parsimonious model minimizes the lags on the independent variable as well as the lags included on the instruments} that complies with the Arellano-Bond and Sargan-Hansen tests. Specifically, we start with one lag of the dependent variable and one valid instrument and use a maximum of three lags of the dependent variable and four valid instruments. We start to evaluate the model with one lag in the independent variable and one instrument. We perform the Arellano-Bond test rejecting the null hypothesis of autocorrelation of order 1 in the first-differenced residuals at the 5\% level of significance; then, we do not reject the null hypothesis of autocorrelation of order 2 in the first-differenced residuals. Concerning the Sargan-Hansen test, we do not reject the null hypothesis of overidentifying restrictions. For the models where the Arellano-Bond test fails, it is necessary to add more lags to the dependent variable. With respect to the Sargan-Hansen test, we add more lags to the variables used as instruments. We increase the number of instruments for a given number of lags of the independent variable; after we exhaust the valid instruments, we increase the number of lags and start the process again until we find a model that passes all tests. If there is a model where we cannot find a valid specification, we reduce the significance level to 1\% and repeat. 