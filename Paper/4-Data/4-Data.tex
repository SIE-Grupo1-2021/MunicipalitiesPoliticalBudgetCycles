\section{Data}

We combine a new collection of administrative data to quantify the effects of  election year on the municipal expenditures. Look at Tables \ref{table:expenditures} - \ref{table:polmayor} for complete description of the variables. 

\paragraph{Expenditures of municipalities} 

The administrative data of the municipalities' expenditures are obtained from CGR. The period available in our observations goes from 2006 to 2020. Municipalities report all their spending and incomes to the CGR, which is required since they're handling public funds. They provide the data disaggregated up to three levels: \textit{Partidas} (level 1), \textit{Grupo de subpartidas} (level 2) and \textit{Subpartidas} (level 3). We select variables that are discretional and prone to manipulation, additionally, we compare them with recurrent stable expenditures. Also, after choosing these variables we keep the ones that report positive expenditures across all the municipalities in more than 12 years, i.e. we drop the expenditures that have more than 243 zero reports. 

Below are described the sources of control variables. 
\paragraph{Demographic and Economic Variables per Municipality}
We have two sources for these variables: CCSS and CGR. CCSS is in charge of registering the actuarial statistics per municipality. This dataset registers population, number of schools, mortality, births, among other variables related to demography. 

\paragraph{Political Context and Personal Characteristics of Mayors}
Mayors are public servants and they're subject to scrutiny from their citizens. That's why they must submit their curriculum vitae when postulating for these positions. We compile public personal information of mayor postulants from their submitted CVs and the Supreme Election Court registry of every Costa Rican citizen civil information. 

%\paragraph{National Macroeconomic Variables}
%These are standard national macroeconomic time series. The BCCR calculates and reports them in various frequencies. In our case, these are annual series. 
