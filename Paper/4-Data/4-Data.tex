\section{Data}\label{sec:data}
We combine a new collection of administrative data to quantify the effects of the election year on municipal expenditures. 
\paragraph{Expenditures of municipalities} 
Table \ref{table:expenditures} presents the administrative data of the municipalities' expenditures are obtained from CGR. The period available in our observations goes from 2006 to 2020. Municipalities report all their spending and income to the CGR, which is required since they are handling public funds. They provide data disaggregated into three levels: \textit{Partidas} (level 1), \textit{Grupo de subpartidas} (level 2), and \textit{Subpartidas} (level 3). We select variables that are discretional and prone to manipulation, additionally, we compare them with recurrent stable expenditures. Also, after choosing these variables, we keep those that report positive expenditures across all municipalities in more than 12 years, that is, we drop the expenditures that have more than 243 zero reports (20\% out of the total observations). 

Then, Table \ref{table:summ2014} provides descriptive statistics of real expenditures per capita in Costa Rican municipalities. There's heterogeneity in spending decisions among municipalities. Also, look that most of the expenditures have positive asymmetries in their distributions. This could respond to their differences in geographical, demographic, socioeconomic, mayor characteristics, and perceived necessities of each municipality.  

\begin{table}[h]
\begin{center}
\caption{Summary of per capita expenditures, 2014\\(Thousands of real 2020 colones)}
\resizebox{\textwidth}{!}{\begin{tabular}{lrrrrrrrrr}
\hline
\hline
Expenditure & Mean & St Dev & 10th & 25th & 50th & 75th & 90th & 95th & 99th \\
\hline
Remunerations & 31.73 & 18.34 & 15.36 & 19.12 & 26.80 & 39.70 & 50.76 & 56.05 & 99.48 \\
Services & 13.42 & 11.87 & 5.47 & 6.79 & 10.86 & 15.27 & 21.26 & 23.94 & 73.13 \\
Durable Goods & 15.45 & 18.19 & 2.47 & 4.23 & 10.17 & 20.27 & 34.99 & 45.90 & 71.56 \\
Basic remuneration & 15.90 & 8.66 & 8.11 & 10.28 & 13.58 & 19.78 & 24.49 & 30.74 & 45.64 \\
Contingent remuneration & 1.47 & 0.96 & 0.60 & 0.93 & 1.18 & 1.75 & 2.54 & 2.90 & 5.32 \\
Rentals & 1.63 & 2.59 & 0.22 & 0.34 & 0.86 & 1.89 & 3.35 & 4.76 & 9.52 \\
Financial and commercial services & 0.80 & 0.92 & 0.17 & 0.30 & 0.58 & 0.93 & 1.47 & 1.94 & 4.97 \\
Training and Protocol & 0.57 & 0.59 & 0.08 & 0.15 & 0.41 & 0.77 & 1.28 & 1.78 & 2.69 \\
Maintenance and repairs & 1.66 & 1.60 & 0.41 & 0.66 & 1.19 & 1.94 & 3.16 & 4.67 & 7.83 \\
Machinery, equipment and mobiliary adquisitions & 3.35 & 4.65 & 0.51 & 0.99 & 2.16 & 3.80 & 6.67 & 11.88 & 19.23 \\
Construction, additions and remodelating & 11.48 & 16.20 & 0.94 & 2.24 & 6.37 & 16.47 & 29.50 & 33.99 & 57.84 \\
Salaries & 14.22 & 7.90 & 7.09 & 8.88 & 12.48 & 17.71 & 22.25 & 26.76 & 42.28 \\
Overtime & 0.56 & 0.49 & 0.09 & 0.23 & 0.46 & 0.75 & 1.10 & 1.43 & 2.23 \\
Subsistence allowance & 0.77 & 0.37 & 0.36 & 0.48 & 0.75 & 0.98 & 1.26 & 1.42 & 1.93 \\
Machines, equipment and mobiliary rentals & 1.19 & 1.49 & 0.04 & 0.21 & 0.46 & 1.74 & 3.07 & 3.92 & 6.75 \\
Advertisement and Publicity & 0.08 & 0.21 & 0.00 & 0.01 & 0.04 & 0.08 & 0.14 & 0.20 & 1.36 \\
Protocol and social activities & 0.39 & 0.45 & 0.01 & 0.08 & 0.22 & 0.51 & 1.04 & 1.30 & 1.61 \\
Buildings, constructions and lands maintenance & 0.21 & 0.42 & 0.00 & 0.01 & 0.04 & 0.18 & 0.62 & 1.19 & 1.80 \\
Roads & 5.57 & 12.14 & 0.00 & 0.55 & 2.11 & 5.50 & 13.63 & 17.37 & 40.92 \\
\hline
Total expenses & 79.26 & 50.02 & 39.14 & 48.43 & 68.16 & 90.81 & 115.31 & 145.13 & 282.77 \\
\hline
\hline
\end{tabular}}
\label{table:summ2014}
\end{center}
\footnotesize
\textit{Note: Table \ref{table:summ2014} presents descriptive statistics of the expenditures per capita in Costa Rican municipalities in 2014. The expenditures' descriptive statistics are reported in CPI-deflated 2020 colones. The statistics are computed using 81 municipalities. We use municipality's population reported each year by CCSS. In particular, we show the mean, standard deviation, and percentiles. This information was extracted from the Comptroller General of the Republic.}
\end{table}


Then, Table \ref{table:sharessumm2014} presents summary statistics for expenditures as a share of total expenses in 2014 including every municipality. Surprisingly, the shares tend to be more symmetric. Remember that we include several levels of disaggregation in Table \ref{table:summ2014} and \ref{table:sharessumm2014}. 

\begin{table}[h]
\begin{center}
\caption{Share of expenditures (\%), 2014}
\resizebox{\textwidth}{!}{\begin{tabular}{lrrrrrrrrr}
\hline
\hline
Expenditure & Mean & St Dev & 10th & 25th & 50th & 75th & 90th & 95th & 99th \\
\hline
Remunerations & 41.43 & 9.40 & 27.77 & 34.04 & 43.39 & 48.61 & 52.49 & 56.15 & 59.86 \\
Services & 16.81 & 6.41 & 9.95 & 11.74 & 15.78 & 20.27 & 25.36 & 26.72 & 33.47 \\
Durable Goods & 17.42 & 12.06 & 4.48 & 6.96 & 13.70 & 25.79 & 36.07 & 39.72 & 44.22 \\
Basic remuneration & 21.14 & 5.50 & 13.77 & 17.48 & 21.32 & 24.30 & 28.31 & 29.57 & 32.21 \\
Contingent remuneration & 1.95 & 0.80 & 1.11 & 1.32 & 1.78 & 2.48 & 3.02 & 3.49 & 3.94 \\
Rentals & 1.96 & 1.91 & 0.28 & 0.60 & 1.30 & 3.03 & 4.24 & 6.41 & 7.30 \\
Financial and commercial services & 0.99 & 0.60 & 0.25 & 0.47 & 0.99 & 1.40 & 1.77 & 1.96 & 2.34 \\
Training and Protocol & 0.67 & 0.50 & 0.10 & 0.30 & 0.59 & 0.92 & 1.42 & 1.61 & 2.01 \\
Maintenance and repairs & 2.28 & 2.86 & 0.78 & 1.11 & 1.71 & 2.57 & 3.01 & 5.12 & 14.07 \\
Machinery, equipment and mobiliary adquisitions & 4.08 & 5.11 & 1.12 & 1.62 & 2.80 & 4.25 & 8.02 & 9.69 & 26.33 \\
Construction, additions and remodelating & 12.75 & 11.12 & 1.63 & 3.22 & 9.52 & 19.27 & 29.17 & 33.48 & 41.52 \\
Salaries & 18.94 & 5.11 & 13.02 & 14.90 & 18.73 & 22.01 & 25.01 & 25.84 & 30.90 \\
Overtime & 0.75 & 0.55 & 0.20 & 0.36 & 0.56 & 1.06 & 1.58 & 1.76 & 2.19 \\
Subsistence allowance & 1.08 & 0.46 & 0.62 & 0.75 & 0.95 & 1.35 & 1.69 & 1.94 & 2.51 \\
Machines, equipment and mobiliary rentals & 1.60 & 1.85 & 0.05 & 0.27 & 0.85 & 2.30 & 3.75 & 5.50 & 7.30 \\
Advertisement and Publicity & 0.09 & 0.10 & 0.00 & 0.01 & 0.06 & 0.12 & 0.21 & 0.27 & 0.44 \\
Protocol and social activities & 0.47 & 0.43 & 0.02 & 0.14 & 0.35 & 0.67 & 1.06 & 1.51 & 1.66 \\
Buildings, constructions and lands maintenance & 0.24 & 0.48 & 0.00 & 0.01 & 0.07 & 0.27 & 0.55 & 0.93 & 2.13 \\
Roads & 6.14 & 7.93 & 0.00 & 0.64 & 3.39 & 9.40 & 13.91 & 28.73 & 31.24 \\
\hline
\hline
\end{tabular}}
\label{table:sharessumm2014}
\end{center}
\footnotesize
\textit{Note: Table \ref{table:sharessumm2014} presents the descriptive statistics of the distribution of expenditures as a share of total expenditure for 2014. The statistics are computed using 81 municipalities. In particular, we show the mean, standard deviation, and percentiles. This information was obtained from the Comptroller General of the Republic.}
\end{table}


The sources of control variables are described below. 
\paragraph{Demographic and Economic Variables per Municipality}
We have two sources for these variables: CCSS and CGR. CCSS is in charge of registering the actuarial statistics per municipality. This data set registers population, number of schools, mortality, births, among other variables related to demography, which allows us to isolate population dynamics to identify the effect of the electoral year. Table \ref{table:demoecon} provides further information. 


\paragraph{Political Context and Personal Characteristics of Mayors}
Mayors are public servants and subject to the scrutiny of their citizens. Therefore, they must submit their curriculum vitae when postulating for these positions. We compile public personal information of mayor postulants from their submitted CVs and the Supreme Election Court registry of every Costa Rican citizen civil information such as age, gender, incumbent's advantage, and type of party. Table \ref{table:polmayor} describes the variables. 


%\paragraph{National Macroeconomic Variables}
%These are standard national macroeconomic time series. The BCCR calculates and reports them in various frequencies. In our case, these are annual series. 
